\documentclass{article}
\usepackage[citecolor=bleu]{hyperref}
\usepackage{amsmath,amsthm,amssymb}
\newtheorem{lemma}{Lemma}
\usepackage{xcolor}
\input{../slides/notations}
\usepackage[natbib=true,backend=biber,citestyle=authoryear]{biblatex}
%\usepackage[natbib=true,style=authoryear,maxcitenames=2]{biblatex}
\bibliography{../bib/stats,../bib/learning}

\title{BML: exercise sheet}
\date{}

\begin{document}
\maketitle

Stars indicate the difficulty level, from 1 to 3. One star means that everyone should be able to do it without too much effort.

\section{Lecture 1}

\subsection{Conjugate priors 101: Gaussians $(\star)$}
\label{s:gaussianConjugacy}
Let $y\vert\mu \sim \cN(\mu,I_N)$ and $\mu\sim \cN(0,a I_N)$, for some $a>0$. Show that
\begin{equation}
  \mu\vert x \sim \cN(b y,bI_N), \text{ where } b=a/(a+1).
  \label{e:gaussianConjugacy}
\end{equation}

\subsection{A conjugate prior on probability vectors $(\star)$}
Let
$$
\Delta_d = \{\theta\in\mathbb[0,1]^d \text{ such that } \sum_{k=1}^d \theta_d = 1\}.
$$
Let further $\alpha\in(\mathbb{R}_+)^d$. The Dirichlet pdf is defined by
 $$
 \text{Dir}(\theta\vert \alpha) = \frac{1}{B(\alpha)} \prod_{k=1}^d \theta_k^{\alpha_k -1} 1_{\theta\in \Delta_d},$$
where
 $ B(\alpha) = \prod_{k=1}^d \Gamma(\alpha_k) / \Gamma(\sum_{k=1}^d \alpha_k)$
 is the so-called beta function. To what likelihood is the Dirichlet conjugate? What is the posterior?

 \subsection{Estimating the mean of a Gaussian $(\star\star)$}
 Let $\mu =(\mu_1,\dots,\mu_N)\in \mathbb{R}^N$, and consider $N$ i.i.d. real variables $y_i\vert \mu \sim \cN(\mu_i, 1)$. We wish to infer $\mu$.
\begin{enumerate}
\item What is the maximum likelihood estimator $\hat\mu_{\text{MLE}}$?
\item Henceforth, we judge estimators by the square loss. The frequentist risk of an estimator $\hat\mu$ is
 $$ R(\hat\mu) = \mathbb{E}_{y\vert\mu}  \Vert \mu - \hat\mu\Vert^2.$$
 show that $R(\hat\mu_{\text{MLE}}) = N$.
\item Suppose we have prior belief that $\mu$ lies near $0$, and we choose to represent it by $\mu\sim \cN(0,aI_N),$ $a>0$. What is the Bayes estimator $\hat\mu_{\text{Bayes}}$? What is its (frequentist) risk $R(\hat\mu_{\text{Bayes}})$? What is its Bayes risk $\mathbb{E}_\mu R(\hat\mu_{\text{Bayes}})$?
\item Since we actually have no idea what $a$ should be, we propose to estimate it from data using empirical Bayes. Show that the marginal of $y$ is
$$
\int p(y,\mu)\d \mu = \cN (y\vert 0,(a+1)I_N).
$$
In particular, what is the law of $S= \Vert y\Vert^2$? Deduce from it that $(N-2)/S$ is an unbiased estimator of $a+1$, and consider the empirical Bayes estimator $$\hat\mu_\text{EB} = \left(1- \frac{N-2}{S}\right)y.$$
What is its Bayes risk?
\item (harder, but elementary; do this only if you have solved all the preceding exercises; see \cite[Section 1.2]{Efr10} for a solution) Show that for $N\geq 3$, for every $\mu\in\mathbb{R}^N$,
\begin{equation}
R(\hat\mu_\text{EB}) < R(\hat\mu_\text{MLE}).
\label{e:stein}
\end{equation}
Frequentists say that $\hat\mu_\text{EB}$ dominates $\mu_\text{MLE}$, in the sense that whatever the value of $\mu$, the risk of $\hat\mu_\text{EB}$ is the smallest of the two. This happens even when $\mu$ is far from zero, in which case one might have thought that our $\cN(0,aI_N)$ prior would have been a poor choice. Finally, if you are a strict Waldian, you should thus prefer $\hat\mu_\text{EB}$ to $\hat\mu_\text{MLE}$. Many people still use $\hat\mu_\text{MLE}$; see \cite[Section 1.3]{Efr10} for a tentative answer.

Equation~\ref{e:stein} is called the James-Stein effect, and is a standard example of why following Bayesian guidelines can end up giving good frequentist estimators. Shrinkage like in $\hat\mu_\text{EB}$ are now commonplace in large-dimensional, penalized regression. For more on frequentist guarantees for Bayesian estimators and shrinkage, see \citep[Sections 7, 8, 9]{PaIn09}.
\end{enumerate}

 \subsection{For more exercises on Bayesian derivations}
 \begin{itemize}
   \item Exercises 5.1 to 5.4 of \citep[Chapter 5]{Mur12}.
   \item Go through Sections 4.4 to 4.6 of \citep{Mur12} with pen and paper. Linear Gaussian models appear all the time.
   \item Exercises 2.6, 2.9, 2.10, 2.13, 2.14, and 2.15 of \citep{MaRo07}. Solutions are here.
 \end{itemize}

\printbibliography

\end{document}
